\documentclass{emulateapj}

\usepackage{color}
\usepackage{url}
\usepackage{graphicx}
%\graphicspath{{figures/}}

\usepackage{listings}
\definecolor{lbcolor}{rgb}{0.9,0.9,0.9}
\lstset{language=Python,
        %basicstyle=\footnotesize\ttfamily,
        showspaces=false,
        showstringspaces=false,
        tabsize=2,
        breaklines=false,
        breakatwhitespace=true,
        identifierstyle=\ttfamily,
        keywordstyle=\bfseries\color[rgb]{0.133,0.545,0.133},
        commentstyle=\color[rgb]{0.133,0.545,0.133},
        stringstyle=\color[rgb]{0.627,0.126,0.941},
    }

\begin{document}

\title{\MakeLowercase{\lstinline{cluster-lensing}}: a \lstinline{P}\MakeLowercase{\lstinline{ython}} Package for Galaxy Clusters and Miscentering}
\author{
Jes Ford\altaffilmark{1},  
Jake VanderPlas\altaffilmark{1}
}

\altaffiltext{1}{University of Washington, eScience Institute, 3910 15th Ave NE, Seattle, WA 98195, USA}


\begin{abstract}
We describe a new open source package for calculating properties of galaxy clusters, including NFW halo profiles with and without the effects of cluster miscentering. This pure-\lstinline{Python} package, \lstinline{cluster-lensing}, provides well-documented and easy-to-use classes and functions for calculating cluster scaling relations, including mass-richness and mass-concentration relations from the literature, as well as the surface mass density $\Sigma(R)$ and differential surface mass density $\Delta\Sigma(R)$ profiles, probed by weak lensing magnification and shear, respectively. Galaxy cluster miscentering is especially a concern for stacked weak lensing shear studies of galaxy clusters, where offsets between the assumed and the true underlying matter distribution
This software has been developed and released in a public GitHub repository, and is licensed under the permissive free MIT license. The \lstinline{cluster-lensing} package can be downloaded either from GitHub, at \url{https://github.com/jesford/cluster-lensing}, or through the Python Package Index, \url{https://pypi.python.org/pypi/cluster-lensing}. Full documentation is available at \url{http://jesford.github.io/cluster-lensing/}.
\end{abstract}

\keywords{gravitational lensing: weak, galaxies: clusters: general, dark matter}

\setcounter{section}{0}
\setcounter{subsection}{0}
\setcounter{subsubsection}{0}

\section{Introduction}
\label{intro}

\begin{itemize} \itemsep -2pt
\item Background about clusters and weak lensing.
\item Existing code? 
\item composite-NFW fits for weak lensing \citep{Ford12, Ford14, Ford15}
\item What is new = miscentering \citep{Johnston07, George12, Ford14, Ford15}
\end{itemize}

\section{Description of the Code}
\label{code}

\begin{itemize} \itemsep -2pt
\item Purpose and general use.
\item \lstinline{SurfaceMassDensity()} class, generic to all NFW halos
\item \lstinline{ClusterEnsemble()} class
\item mass-richness functions
\item mass-concentration functions
\end{itemize}

\section{Examples}
\label{ex}

\begin{itemize} \itemsep -2pt
\item No miscentering
\item With miscentering
\item others...
\end{itemize}

\section{Future Development}
\label{future}

Plans for the future.

\section{Summary}
\label{summary}

Summary goes here.

%\vspace{10.pt}

\section*{Acknowledgements}
The authors are grateful for funding from the Washington Research Foundation Fund for Innovation in Data-Intensive Discovery and the Moore/Sloan Data Science Environments Project at the University of Washington.


\bibliographystyle{apj}
\bibliography{References}

\end{document}
