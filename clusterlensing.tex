\documentclass{emulateapj}

\usepackage{color}
\usepackage{graphicx}
\usepackage[backref,breaklinks,colorlinks,citecolor=blue]{hyperref}
\usepackage[all]{hypcap}
\renewcommand*{\backref}[1]{[#1]}
%\usepackage{url}
%\graphicspath{{figures/}}

\usepackage{listings}
\definecolor{lbcolor}{rgb}{0.9,0.9,0.9}
\lstset{language=Python,
        %basicstyle=\footnotesize\ttfamily,
        showspaces=false,
        showstringspaces=false,
        tabsize=2,
        breaklines=false,
        breakatwhitespace=true,
        identifierstyle=\ttfamily,
        keywordstyle=\bfseries\color[rgb]{0.133,0.545,0.133},
        commentstyle=\color[rgb]{0.133,0.545,0.133},
        stringstyle=\color[rgb]{0.627,0.126,0.941},
    }

\begin{document}

\title{\MakeLowercase{\lstinline{cluster-lensing}}: a \lstinline{P}\MakeLowercase{\lstinline{ython}} Package for Galaxy Clusters and Miscentering}
\author{
Jes Ford\altaffilmark{1},  
Jake VanderPlas\altaffilmark{1}
}

\altaffiltext{1}{eScience Institute, University of Washington}


\begin{abstract}
We describe a new open source package for calculating properties of galaxy clusters, including NFW halo profiles with and without the effects of cluster miscentering. This pure-\lstinline{Python} package, \lstinline{cluster-lensing}, provides well-documented and easy-to-use classes and functions for calculating cluster scaling relations, including mass-richness and mass-concentration relations from the literature, as well as the surface mass density $\Sigma(R)$ and differential surface mass density $\Delta\Sigma(R)$ profiles, probed by weak lensing magnification and shear, respectively. Galaxy cluster miscentering is especially a concern for stacked weak lensing shear studies of galaxy clusters, where offsets between the assumed and the true underlying matter distribution
This software has been developed and released in a public GitHub repository, and is licensed under the permissive free MIT license. The \lstinline{cluster-lensing} package can be downloaded through the Python Package Index, \url{https://pypi.python.org/pypi/cluster-lensing}, or directly from GitHub, at \url{https://github.com/jesford/cluster-lensing}. Full documentation is available at \url{http://jesford.github.io/cluster-lensing/}.
\end{abstract}

\keywords{gravitational lensing: weak, galaxies: clusters: general, dark matter, methods: data analysis, methods: numerical}

\setcounter{section}{0}
\setcounter{subsection}{0}
\setcounter{subsubsection}{0}

\section{Introduction}
\label{intro}

Clusters of galaxies are the largest gravitationally collapsed structures to have formed in the history of the universe.  As such, they are interesting both from a cosmological as well as an astrophysics perspective. In the former case, the galaxy cluster number density as a function of mass, known as the cluster mass function, is a probe of cosmological parameters including the fractional matter density $\Omega_{\rm M}$ and the normalization of the matter power spectrum $\sigma_8$. Astrophysically, the deep potential wells of galaxy clusters are environments useful for testing theories of general relativity, galaxy evolution, and gas and plasma physics, among other things \citep{Voit05}. 

The common thread among these diverse investigations is the requisite knowledge of the mass of the galaxy cluster, which is largely composed of its invisible dark matter halo. Although many techniques exist for estimated the total mass of these systems, weak lensing has emerged as somewhat of a gold standard, since it is sensitive to the mass itself, and not to the dynamical state or other biased tracers of the underlying mass. Since weak lensing masses are often considered the "true" masses, against which other estimates are compared \citep[e.g.]{Leauthaud10, vonderLinden14, Hoekstra15}, it is paramount that cluster masses from weak lensing modeling are as unbiased as possible.

\begin{itemize} \itemsep -2pt
\item Background about clusters and weak lensing.
\item NFW halos \citep{nfw97, Wright00}
\item composite-NFW fits for weak lensing \citep{Ford12, Ford14, Ford15}
\item What is new = miscentering \citep{Johnston07, George12, Ford14, Ford15}
\end{itemize}


\section{Description of the Code}
\label{code}

\begin{itemize} \itemsep -2pt
\item Purpose and general use.
\item Relation to existing code
\item \lstinline{SurfaceMassDensity()} class, generic to all NFW halos
\item \lstinline{ClusterEnsemble()} class
\item mass-richness functions
\item mass-concentration functions
\item We use units from the \lstinline{astropy.units} package \citep{astropy13}.
\end{itemize}

\section{Examples}
\label{ex}

\begin{itemize} \itemsep -2pt
\item No miscentering
\item With miscentering
\item others...
\end{itemize}

\section{Future Development}
\label{future}

Plans for the future.

\section{Summary}
\label{summary}

Summary goes here.

%\vspace{10.pt}

\section*{Acknowledgements}
The authors are grateful for funding from the Washington Research Foundation Fund for Innovation in Data-Intensive Discovery and the Moore/Sloan Data Science Environments Project at the University of Washington. This research made use of Astropy, a community-developed core Python package for Astronomy (Astropy Collaboration, 2013), \url{http://www.astropy.org}.


\bibliographystyle{apj}
\bibliography{References}

\end{document}
